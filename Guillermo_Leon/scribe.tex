%
% This is the LaTeX template file for lecture notes for EE 382C/EE 361C.
%
% To familiarize yourself with this template, the body contains
% some examples of its use.  Look them over.  Then you can
% run LaTeX on this file.  After you have LaTeXed this file then
% you can look over the result either by printing it out with
% dvips or using xdvi.
%
% This template is based on the template for Prof. Sinclair's CS 270.

\documentclass[twoside]{article}
\usepackage{graphics}
\usepackage{multirow}
\usepackage[table,xcdraw]{xcolor}
\setlength{\oddsidemargin}{0.25 in}
\setlength{\evensidemargin}{-0.25 in}
\setlength{\topmargin}{-0.6 in}
\setlength{\textwidth}{6.5 in}
\setlength{\textheight}{8.5 in}
\setlength{\headsep}{0.75 in}
\setlength{\parindent}{0 in}
\setlength{\parskip}{0.1 in}

%
% The following commands set up the lecnum (lecture number)
% counter and make various numbering schemes work relative
% to the lecture number.
%
\newcounter{lecnum}
\renewcommand{\thepage}{\thelecnum-\arabic{page}}
\renewcommand{\thesection}{\thelecnum.\arabic{section}}
\renewcommand{\theequation}{\thelecnum.\arabic{equation}}
\renewcommand{\thefigure}{\thelecnum.\arabic{figure}}
\renewcommand{\thetable}{\thelecnum.\arabic{table}}

%
% The following macro is used to generate the header.
%
\newcommand{\lecture}[4]{
   \pagestyle{myheadings}
   \thispagestyle{plain}
   \newpage
   \setcounter{lecnum}{#1}
   \setcounter{page}{1}
   \noindent
   \begin{center}
   \framebox{
      \vbox{\vspace{2mm}
    \hbox to 6.28in { {\bf EE 382C/361C: Multicore Computing
                        \hfill Fall 2016} }
       \vspace{4mm}
       \hbox to 6.28in { {\Large \hfill Lecture #1: #2  \hfill} }
       \vspace{2mm}
       \hbox to 6.28in { {\it Lecturer: #3 \hfill Scribe: #4} }
      \vspace{2mm}}
   }
   \end{center}
   \markboth{Lecture #1: #2}{Lecture #1: #2}
   %{\bf Disclaimer}: {\it These notes have not been subjected to the
   %usual scrutiny reserved for formal publications.  They may be distributed
   %outside this class only with the permission of the Instructor.}
   \vspace*{4mm}
}

%
% Convention for citations is authors' initials followed by the year.
% For example, to cite a paper by Leighton and Maggs you would type
% \cite{LM89}, and to cite a paper by Strassen you would type \cite{S69}.
% (To avoid bibliography problems, for now we redefine the \cite command.)
% Also commands that create a suitable format for the reference list.
\renewcommand{\cite}[1]{[#1]}
\def\beginrefs{\begin{list}%
        {[\arabic{equation}]}{\usecounter{equation}
         \setlength{\leftmargin}{2.0truecm}\setlength{\labelsep}{0.4truecm}%
         \setlength{\labelwidth}{1.6truecm}}}
\def\endrefs{\end{list}}
\def\bibentry#1{\item[\hbox{[#1]}]}

%Use this command for a figure; it puts a figure in wherever you want it.
%usage: \fig{NUMBER}{SPACE-IN-INCHES}{CAPTION}
\newcommand{\fig}[3]{
			\vspace{#2}
			\begin{center}
			Figure \thelecnum.#1:~#3
			\end{center}
	}
% Use these for theorems, lemmas, proofs, etc.
\newtheorem{theorem}{Theorem}[lecnum]
\newtheorem{lemma}[theorem]{Lemma}
\newtheorem{proposition}[theorem]{Proposition}
\newtheorem{claim}[theorem]{Claim}
\newtheorem{corollary}[theorem]{Corollary}
\newtheorem{definition}[theorem]{Definition}
\newenvironment{proof}{{\bf Proof:}}{\hfill\rule{2mm}{2mm}}

% **** IF YOU WANT TO DEFINE ADDITIONAL MACROS FOR YOURSELF, PUT THEM HERE:

\begin{document}
%FILL IN THE RIGHT INFO.
%\lecture{**LECTURE-NUMBER**}{**DATE**}{**LECTURER**}{**SCRIBE**}
\lecture{1}{November 16}{Vijay Garg}{Guillermo Le\'on}
%\footnotetext{These notes are partially based on those of Nigel Mansell.}

% **** YOUR NOTES GO HERE:

% Some general latex examples and examples making use of the
% macros follow.  
%**** IN GENERAL, BE BRIEF. LONG SCRIBE NOTES, NO MATTER HOW WELL WRITTEN,
%**** ARE NEVER READ BY ANYBODY.
\section{Chapter 6 Exercises}

\begin{enumerate}
    \setcounter{enumi}{9}
    \item Given payoff matrix.
    
    \begin{center}
        \begin{tabular}{llll}
                                  &                        & \multicolumn{2}{l}{Player B}                        \\
                                  &                        & L                        & R                        \\ \cline{3-4} 
        \multirow{2}{*}{Player A} & \multicolumn{1}{l|}{U} & \multicolumn{1}{l|}{3,3} & \multicolumn{1}{l|}{1,2} \\ \cline{3-4} 
                                  & \multicolumn{1}{l|}{D} & \multicolumn{1}{l|}{2,1} & \multicolumn{1}{l|}{3,0} \\ \cline{3-4} 
        \end{tabular}    
    \end{center}
    
    \begin{enumerate}
        \item All pure strategy Nash equilibria of this game. \\
        
        Let's try first finding dominant strategies. In this game we can see that Player B has $L$ as a strictly dominant strategy since $P_B(U, L) > P_B(U, R)$ and $P_B(D, L) > P_B(D, R)$. In other words, strategy $L$ is always the strict best response to whichever strategy Player A chooses. Because this is the case, it is expected that Player B will go with $L$, therefore the best response from Player A to $L$ is $U$ because it maximizes her payoff. The only pure strategy Nash equilibrium is $(U, L)$. \\
        
        \begin{center}
            \begin{tabular}{llll}
                                       &                                 & \multicolumn{2}{c}{Player B}                                                      \\
                                       &                                 & \multicolumn{1}{c}{\textit{L}}                   & \multicolumn{1}{c}{\textit{R}} \\ \cline{3-4} 
                                       & \multicolumn{1}{l|}{\textit{U}} & \multicolumn{1}{l|}{\cellcolor[HTML]{FFCCC9}3,3} & \multicolumn{1}{l|}{1,2}       \\ \cline{3-4} 
            \multirow{-2}{*}{Player A} & \multicolumn{1}{l|}{\textit{D}} & \multicolumn{1}{l|}{2,1}                         & \multicolumn{1}{l|}{3,0}       \\ \cline{3-4} 
            \end{tabular}    
        \end{center}
        
        \item Changing player A's payoff from $(U, L)$ to not have pure-strategy Nash equilibrium. \\
        
        \emph{No}, as it was stated in the answer above, Player B is indifferent of the payoff of Player A. In the original case Player A does not have a dominant strategy, let's say that we modify $P_A(U, L) = 1$. In this case Player A now has a strictly dominant strategy $D$, however the best response from Player B is still to go with $L$. This only moves the Nash equilibrium from $(U, L)$ to $(D, L)$. 
        
        \begin{center}
            \begin{tabular}{llll}
                                       &                                 & \multicolumn{2}{c}{Player B}                                                      \\
                                       &                                 & \multicolumn{1}{c}{\textit{L}}                   & \multicolumn{1}{c}{\textit{R}} \\ \cline{3-4} 
                                       & \multicolumn{1}{l|}{\textit{U}} & \multicolumn{1}{l|}{\textbf{1},3}                & \multicolumn{1}{l|}{1,2}       \\ \cline{3-4} 
            \multirow{-2}{*}{Player A} & \multicolumn{1}{l|}{\textit{D}} & \multicolumn{1}{l|}{\cellcolor[HTML]{FFCCC9}2,1} & \multicolumn{1}{l|}{3,0}       \\ \cline{3-4} 
            \end{tabular}
        \end{center}
        
        Finally let's suppose we change the payoff $P_A(D, L) = 3$, in this case we now have 2 Nash equilibria because $U$ and $D$ are both best responses to $L$ from player B. \\
        
        \begin{center}
            \begin{tabular}{llll}
                                       &                                 & \multicolumn{2}{c}{Player B}                                                               \\
                                       &                                 & \multicolumn{1}{c}{\textit{L}}                            & \multicolumn{1}{c}{\textit{R}} \\ \cline{3-4} 
                                       & \multicolumn{1}{l|}{\textit{U}} & \multicolumn{1}{l|}{\cellcolor[HTML]{FFCCC9}3,3}          & \multicolumn{1}{l|}{1,2}       \\ \cline{3-4} 
            \multirow{-2}{*}{Player A} & \multicolumn{1}{l|}{\textit{D}} & \multicolumn{1}{l|}{\cellcolor[HTML]{FFCCC9}\textbf{3},1} & \multicolumn{1}{l|}{3,0}       \\ \cline{3-4} 
            \end{tabular}
        \end{center}

        \item Changing player B's payoff from (\textsl{U, L}) to not have pure-strategy Nash equilibrium. \\

        \emph{Yes}, since Player B has a strictly dominant strategy it is just a matter of removing it. If we set the payoff for player B for the pair of strategies $U$ and $L$ to 1 or 0. In this case neither of the players have a dominant strategy. Let's analyze then the best responses for each player and each strategy.
        Player A chooses $U$, then best response from player B is $R$, but if player B chooses $R$ then best response from player A is $D$. Finally if player A chooses $D$, then the best response from B is $L$ and when player B chooses $L$, the best response from player A is $U$, which is the initial strategy. Therefore there is no pure strategy Nash equilibrium.
        
        \begin{center}
            \begin{tabular}{llll}
                                      &                                 & \multicolumn{2}{c}{Player B}                                       \\
                                      &                                 & \multicolumn{1}{c}{\textit{L}}    & \multicolumn{1}{c}{\textit{R}} \\ \cline{3-4} 
            \multirow{2}{*}{Player A} & \multicolumn{1}{l|}{\textit{U}} & \multicolumn{1}{l|}{3,\textbf{1}} & \multicolumn{1}{l|}{1,2}       \\ \cline{3-4} 
                                      & \multicolumn{1}{l|}{\textit{D}} & \multicolumn{1}{l|}{2,1}          & \multicolumn{1}{l|}{3,0}       \\ \cline{3-4} 
            \end{tabular}
        \end{center}
        
        
    \end{enumerate}
    
    \item Explain why the strategies used in an equilibrium of this game will not be dominated strategies. \\
    
    Let's recall that strategies in a pure strategy Nash equilibrium are the best responses to each other from both players. More precisely, let's suppose that Player 1 chooses strategy $S$ and Player 2 chooses $T$, then we say that $(S, T)$ are in a Nash equilibrium if $S$ is a best response to $T$, and $T$ is a best response to $S$. Because they are considered best strategies, that means that they produce a payoff that is at least as good as any other strategy. In other words, let's suppose that strategy $S$ is indeed a dominated strategy, this means that there is a strategy $S\prime$ with a greater payoff, therefore $S\prime$ is a best response to a given strategy $T$ and because of this $S$ cannot be part of the Nash equilibrium.
    
\end{enumerate}

\end{document}
