%
% This is the LaTeX template file for lecture notes for EE 382C/EE 361C.
%
% To familiarize yourself with this template, the body contains
% some examples of its use.  Look them over.  Then you can
% run LaTeX on this file.  After you have LaTeXed this file then
% you can look over the result either by printing it out with
% dvips or using xdvi.
%
% This template is based on the template for Prof. Sinclair's CS 270.

\documentclass[twoside]{article}
\usepackage{graphics}
\setlength{\oddsidemargin}{0.25 in}
\setlength{\evensidemargin}{-0.25 in}
\setlength{\topmargin}{-0.6 in}
\setlength{\textwidth}{6.5 in}
\setlength{\textheight}{8.5 in}
\setlength{\headsep}{0.75 in}
\setlength{\parindent}{0 in}
\setlength{\parskip}{0.1 in}
\usepackage[]{algorithm2e}
%
% The following commands set up the lecnum (lecture number)
% counter and make various numbering schemes work relative
% to the lecture number.
%
\newcounter{lecnum}
\renewcommand{\thepage}{\thelecnum-\arabic{page}}
\renewcommand{\thesection}{\thelecnum.\arabic{section}}
\renewcommand{\theequation}{\thelecnum.\arabic{equation}}
\renewcommand{\thefigure}{\thelecnum.\arabic{figure}}
\renewcommand{\thetable}{\thelecnum.\arabic{table}}

%
% The following macro is used to generate the header.
%
\newcommand{\lecture}[4]{
   \pagestyle{myheadings}
   \thispagestyle{plain}
   \newpage
   \setcounter{lecnum}{#1}
   \setcounter{page}{1}
   \noindent
   \begin{center}
   \framebox{
      \vbox{\vspace{2mm}
    \hbox to 6.28in { {\bf EE 382V: Social Computing
                        \hfill Fall 2018} }
       \vspace{4mm}
       \hbox to 6.28in { {\Large \hfill Lecture #1: #2  \hfill} }
       \vspace{2mm}
       \hbox to 6.28in { {\it Lecturer: #3 \hfill Scribe: #4} }
      \vspace{2mm}}
   }
   \end{center}
   \markboth{Lecture #1: #2}{Lecture #1: #2}
   %{\bf Disclaimer}: {\it These notes have not been subjected to the
   %usual scrutiny reserved for formal publications.  They may be distributed
   %outside this class only with the permission of the Instructor.}
   \vspace*{4mm}
}

%
% Convention for citations is authors' initials followed by the year.
% For example, to cite a paper by Leighton and Maggs you would type
% \cite{LM89}, and to cite a paper by Strassen you would type \cite{S69}.
% (To avoid bibliography problems, for now we redefine the \cite command.)
% Also commands that create a suitable format for the reference list.
\renewcommand{\cite}[1]{[#1]}
\def\beginrefs{\begin{list}%
        {[\arabic{equation}]}{\usecounter{equation}
         \setlength{\leftmargin}{2.0truecm}\setlength{\labelsep}{0.4truecm}%
         \setlength{\labelwidth}{1.6truecm}}}
\def\endrefs{\end{list}}
\def\bibentry#1{\item[\hbox{[#1]}]}

%Use this command for a figure; it puts a figure in wherever you want it.
%usage: \fig{NUMBER}{SPACE-IN-INCHES}{CAPTION}
\newcommand{\fig}[3]{
			\vspace{#2}
			\begin{center}
			Figure \thelecnum.#1:~#3
			\end{center}
	}
% Use these for theorems, lemmas, proofs, etc.
\newtheorem{theorem}{Theorem}[lecnum]
\newtheorem{lemma}[theorem]{Lemma}
\newtheorem{proposition}[theorem]{Proposition}
\newtheorem{claim}[theorem]{Claim}
\newtheorem{corollary}[theorem]{Corollary}
\newtheorem{definition}[theorem]{Definition}
\newenvironment{proof}{{\bf Proof:}}{\hfill\rule{2mm}{2mm}}

% **** IF YOU WANT TO DEFINE ADDITIONAL MACROS FOR YOURSELF, PUT THEM HERE:

\begin{document}
%FILL IN THE RIGHT INFO.
%\lecture{**LECTURE-NUMBER**}{**DATE**}{**LECTURER**}{**SCRIBE**}
\lecture{3 Session 2 }{September 28}{Vijay Garg}{Carmina Francia}
%\footnotetext{These notes are partially based on those of Nigel Mansell.}

% **** YOUR NOTES GO HERE:

% Some general latex examples and examples making use of the
% macros follow.  
%**** IN GENERAL, BE BRIEF. LONG SCRIBE NOTES, NO MATTER HOW WELL WRITTEN,
%**** ARE NEVER READ BY ANYBODY.
\section{Stable Matching/Marriage}
Matching has a lot of real world applications. We see it in markets, buyers/sellers, computer processes, residencies, and so on. So far we've used matching in the context of girls likes boys and boys like girls. These matchings can sometimes have \textit{preferences}.  

This is called \textbf{Stable matching/marriage}. It is finding a matching between two equal sets each following an ordered preference list. 

For example, let's  create two preference lists: 
\begin{table}[!htb]
    \begin{minipage}{.5\linewidth}
      \caption{Men Preference List}
      \centering
        \begin{tabular}{c | c c c c}
          \hline
          \hline
            M1 & W4 & W1 & W2 & W3 \\ 
 			M2 & W2 & W3 & W1 & W4 \\  
 			M3 & W2 & W4 & W3 & W1 \\ 
  			M4 & W3 & W1 & W4 & W2 \\ 
        \end{tabular}
    \end{minipage}%
    \begin{minipage}{.5\linewidth}
      \centering
        \caption{Women Preference List}
        \begin{tabular}{c | c c c c}
          \hline
          \hline
            W1 & M4 & M1 & M3 & M2 \\ 
            W2 & M1 & M3 & M2 & M4 \\ 
            W3 & M1 & M2 & M3 & M4 \\ 
            W4 & M4 & M1 & M3 & M2 \\ 
        \end{tabular}
    \end{minipage} 
\end{table}

This is still a bipartite and a good matching would be when we don't  have one woman assigned to two men and vice versa.

\subsection*{Blocking Pair}
What kind of marriage would not last long? 
Let's say we have the following matchings: \textit{(m,w) }and \textit{(m', w')}. 
If \textit{m} and \textit{w'} are not engaged, but m prefers \textit{w'} to w and \textit{w'} prefers \textit{m} to \textit{m'} , then there's an incentive for them to break. This is called a \textbf{blocking pair}. 

\textbf{The goal of stable matching is to find a matching that has \textit{no} blocking pair. }

\section{Gale and Shapley Algorithm}
The idea is similar to DGS algorithm: people bidding and going for the most valuable item. We will have men proposing (bidders). The difference is the other set, in this case women, has a say. Women can reject men. \newline
\begin{algorithm}[H]
 \KwData{men preference list and women preference list}
 \KwResult{(m,w) matchings}
Assuming number of men is equal to number of women\;
 Put all men in a list (initially everyone if free)\;
 \While{free list of men is not empty}{
  pick any free man\;
  man proposes to most preferred woman that he has not proposed to\;
  \eIf{woman is free}{
   men is engaged with that woman\; 
   }{
   woman rejects the less desirable man of the two men based on her preference list\;
  }
  rejected man go back to free list (does not matter where in the list)
 }
\end{algorithm}

\textbf{Key notes:}
\begin{itemize}
\item \textit{engaged} means man and woman are not married. Matchings can change as algorithm progresses.
\item Women rejects based on their preference list
\item if you're a \textit{free} woman, you cannot say no. Only way to reject is if a better man comes.
\item Once a woman is engaged, she cannot be free.
\item Algorithm \textbf{ALWAYS} terminates and everybody is married.
\end{itemize}
This example is \textit{man-optimal}. It is seen as worse from women's perspective. However, as the algorithm progresses, men decrease on their preference while women are safe or their preference increases. 

\section{Stable Matching Example}

Let's walk through a man-optimal example...
\begin{table}[!htb]
    \begin{minipage}{.5\linewidth}
    \caption{Men Preference List}
      \centering
        \begin{tabular}{c | c c c c}
          \hline
          \hline
            \textbf{M1} & \textbf{W4} & W1 & W2 & W3 \\ 
 			\textbf{M2 }& \textit{W2}& \textbf{W3} & W1 & W4 \\  
 			\textbf{M3} &\textbf{ W2} & W4 & W3 & W1 \\ 
  			\textbf{M4} &\textit{ W3 }& \textbf{W1} & W4 & W2 \\ 
        \end{tabular}
    \end{minipage}%
    \begin{minipage}{.5\linewidth}
      \centering
      \caption{Women Preference List}
        \begin{tabular}{c | c c c c}
          \hline
          \hline
            \textbf{W1} & \textbf{M4} & M1 & M3 & M2 \\ 
            \textbf{W2} & \textit{M1} & \textbf{M3} & M2 & M4 \\ 
            \textbf{W3} & \textit{M1} & \textbf{M2} & M3 & M4 \\ 
            \textbf{W4} & \textit{M4} & \textbf{M1} & M3 & M2 \\ 
        \end{tabular}
    \end{minipage} 
\end{table} 

\begin{enumerate}
\item Let's begin with \textit{M1}. \textit{W4} is free so \textit{M1} and \textit{W4} becomes engaged. 
\item We move on to \textit{M2}. \textit{W2} is free and becomes engaged with \textit{M2}.
\item We go to \textit{M3}. \textit{W2} is already engaged, so we check the woman's preference list. \textit{W2} prefers \textit{M3} over \textit{M2}. So now, \textit{M3} and \textit{W2} are engaged. \textit{M2} goes back to the free list.
\item We go to \textit{M4}. \textit{W3} is free and becomes engaged with \textit{M4}.
\item We pick \textit{M2}. The next woman on his preference list that he has not proposed to is \textit{W3}. \textit{W3} is engaged to \textit{M4}. We check the \textit{W3}'s preference list. \textit{W3} prefers \textit{M3} over \textit{M4}. So \textit{M3} and \textit{W3} becomes engaged and \textit{M4} goes back to the list.
\item We pick \textit{M4}. \textit{W1} is free so \textit{M4} and \textit{W1} becomes engaged.
\end{enumerate}
The list is now empty and returns the matchings: \textit{(M1, W4)}, \textit{(M2, W3)}, \textit{(M3, W2)}, \textit{(M4, W1)}. Since we don't see any blocking pairs, this is a stable matching.

Now let's say it's a woman-optimal. Same steps apply except we would initialize with all women in the free list and pick a woman to start with. Once the list is empty and everybody is married/matched, you will notice that we come up with the same matchings:\textit{(W1, M4)}, \textit{(W2, M3)}, \textit{(W3, M2)}, \textit{(W4, M1)}.\newline \newline
\textbf{Note:} If the man-optimal and woman-optimal are \textit{equal}, this is the \textbf{only stable matching}.

\end{document}





