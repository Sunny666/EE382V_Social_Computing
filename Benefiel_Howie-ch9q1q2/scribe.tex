%
% This is the LaTeX template file for lecture notes for EE 382C/EE 361C.
%
% To familiarize yourself with this template, the body contains
% some examples of its use.  Look them over.  Then you can
% run LaTeX on this file.  After you have LaTeXed this file then
% you can look over the result either by printing it out with
% dvips or using xdvi.
%
% This template is based on the template for Prof. Sinclair's CS 270.

\documentclass[twoside]{article}
\usepackage{graphics}
\usepackage{multirow}
\usepackage{xfrac}
\usepackage[table,xcdraw]{xcolor}
\setlength{\oddsidemargin}{0.25 in}
\setlength{\evensidemargin}{-0.25 in}
\setlength{\topmargin}{-0.6 in}
\setlength{\textwidth}{6.5 in}
\setlength{\textheight}{8.5 in}
\setlength{\headsep}{0.75 in}
\setlength{\parindent}{0 in}
\setlength{\parskip}{0.1 in}

%
% The following commands set up the lecnum (lecture number)
% counter and make various numbering schemes work relative
% to the lecture number.
%
\newcounter{lecnum}
\renewcommand{\thepage}{\thelecnum-\arabic{page}}
\renewcommand{\thesection}{\thelecnum.\arabic{section}}
\renewcommand{\theequation}{\thelecnum.\arabic{equation}}
\renewcommand{\thefigure}{\thelecnum.\arabic{figure}}
\renewcommand{\thetable}{\thelecnum.\arabic{table}}

%
% The following macro is used to generate the header.
%
\newcommand{\lecture}[4]{
   \pagestyle{myheadings}
   \thispagestyle{plain}
   \newpage
   \setcounter{lecnum}{#1}
   \setcounter{page}{1}
   \noindent
   \begin{center}
   \framebox{
      \vbox{\vspace{2mm}
    \hbox to 6.28in { {\bf EE 382C Social Computing
                        \hfill Fall 2018} }
       \vspace{4mm}
       \hbox to 6.28in { {\Large \hfill Lecture #1: #2  \hfill} }
       \vspace{2mm}
       \hbox to 6.28in { {\it Chapter 9 \hfill Questions: 1 \& 2} }
      \vspace{2mm}}
   }
   \end{center}
   \markboth{Lecture #1: #2}{Lecture #1: #2}
   %{\bf Disclaimer}: {\it These notes have not been subjected to the
   %usual scrutiny reserved for formal publications.  They may be distributed
   %outside this class only with the permission of the Instructor.}
   \vspace*{4mm}
}

%
% Convention for citations is authors' initials followed by the year.
% For example, to cite a paper by Leighton and Maggs you would type
% \cite{LM89}, and to cite a paper by Strassen you would type \cite{S69}.
% (To avoid bibliography problems, for now we redefine the \cite command.)
% Also commands that create a suitable format for the reference list.
\renewcommand{\cite}[1]{[#1]}
\def\beginrefs{\begin{list}%
        {[\arabic{equation}]}{\usecounter{equation}
         \setlength{\leftmargin}{2.0truecm}\setlength{\labelsep}{0.4truecm}%
         \setlength{\labelwidth}{1.6truecm}}}
\def\endrefs{\end{list}}
\def\bibentry#1{\item[\hbox{[#1]}]}

%Use this command for a figure; it puts a figure in wherever you want it.
%usage: \fig{NUMBER}{SPACE-IN-INCHES}{CAPTION}
\newcommand{\fig}[3]{
			\vspace{#2}
			\begin{center}
			Figure \thelecnum.#1:~#3
			\end{center}
	}
% Use these for theorems, lemmas, proofs, etc.
\newtheorem{theorem}{Theorem}[lecnum]
\newtheorem{lemma}[theorem]{Lemma}
\newtheorem{proposition}[theorem]{Proposition}
\newtheorem{claim}[theorem]{Claim}
\newtheorem{corollary}[theorem]{Corollary}
\newtheorem{definition}[theorem]{Definition}
\newenvironment{proof}{{\bf Proof:}}{\hfill\rule{2mm}{2mm}}

% **** IF YOU WANT TO DEFINE ADDITIONAL MACROS FOR YOURSELF, PUT THEM HERE:

\begin{document}
%FILL IN THE RIGHT INFO.
%\lecture{**LECTURE-NUMBER**}{**DATE**}{**LECTURER**}{**SCRIBE**}
\lecture{1}{November 28}{Vijay Garg}{Howie Benefiel}
%\footnotetext{These notes are partially based on those of Nigel Mansell.}

% **** YOUR NOTES GO HERE:

% Some general latex examples and examples making use of the
% macros follow.
%**** IN GENERAL, BE BRIEF. LONG SCRIBE NOTES, NO MATTER HOW WELL WRITTEN,
%**** ARE NEVER READ BY ANYBODY.
\section{Chapter 9 Questions 1 \& 2}

\begin{enumerate}
    \setcounter{enumi}{0}
    \item The seller will run a sealed-bid, second-price auction. Your firm will bid in the auction, but it does not know for sure how many other bidders will participate in the auction. There will be either two or three other bidders in addition to your firm. All bidders have independent, private values for the good. Your firm's value for the good is c. What bid should your firm submit, and how does it depend on the number of other bidders who show up? \\

\textbf{Answer.} \\
For sealed-bid, second-price auctions, the number of bidders, $ n $ , does not affect what you should bid for $ n \geq 2 $ .
That is because regardless of the number of bidders, you are incentivized to bid your true value for the item.

If you bid a price, $ b_{a} $ , which is above your true value, $ v $ , the only time that would change your realized payoff is if $ b_{a} $ wins when a bid of $ v $ would not have won.
In this case, you would pay your opponent's price, which is above your true value.
By paying that price, you realize a negative payoff because you overpaid for the item.

If you bid a price, $ b_{b} $ , which is below your your true value, the only time the only time this bid would change your realized payoff is if $ b_{b} $ loses when your bid of $ v $ would have won.
In this case, your payoff is zero because you lost when your payoff could have been $ v - b_{o} \geq 0 $ where $ b_{o} $ is your opponent's bid.

In this entire analysis, your bid has only ever been dependent on one other competing bid, so you should bid your true value if there is at least one other bidder.

\item In this problem we will ask how the number of bidders in a second-price, sealed-bid auction affects how much the seller can expect to receive for his object. Assume that there are two bidders who have independent, private values $ v_{i} $ which are either 1 or 3. For each bidder, the probabilities of 1 and 3 are both $ \sfrac{1}{2} $ . (If there is a tie at a bid of $ x $ for the highest bid the winner is selected at random from among the highest bidders and the price is $ x $ .)

\begin{enumerate}

\item Show that the seller's expected revenue is $ \sfrac{6}{4} $ .\\

\textbf{Answer.} \\
In a closed-bid, second-price auction, the bidders are always incentivized to bid their true value.
By case analysis, that means there is a $ \sfrac{1}{4} $ chance both bids are 3, making the revenue 3;
a $ \sfrac{1}{2} $ chance one bid is 1 and the other 3, making the revenue 1 (because 1 is the second price);
and a $ \sfrac{1}{4} $ chance both bids are 1, making the revenue 1.
By multiplying the chances by the revenue of each case and summing, $ \sfrac{1}{4} \cdot 3 + \sfrac{1}{2} \cdot 1 + \sfrac{1}{3} \cdot 1 $, the expected revenue is $ \sfrac{6}{4} $ . \\

\newpage

\item Now let's suppose that there are three bidders who have independent, private values $ v_{i} $ which are either 1 or 3. For each bidder, the probabilities of 1 and 3 are both $ \sfrac{1}{2} $ . What is the seller's expected revenue in this case? \\

\textbf{Answer.} \\
We perform a similar case analysis and find there is a $ \sfrac{1}{8} $ chance all three bids are 3, making the revenue 3;
a \sfrac{3}{8} chance two bids are 3 and one is 1, making the revenue 3;
a \sfrac{3}{8} chance two bids are 1, and one is 3, making the revenue 1;
and a \sfrac{1}{8} chance all three bids are 1, making the revenue 1.
Applying the formula for expected payoff, $ \sfrac{1}{8} \cdot 3 + \sfrac{3}{8} \cdot 3 + \sfrac{3}{8} \cdot 1 + \sfrac{1}{8} \cdot 1 $, the expected revenue is 2. \\

\item Briefly explain why changing the number of bidders affects the seller's expected revenue. \\

\textbf{Answer.} \\
The number of players changes the seller's expected revenue because it becomes more and more likely that two bids will be the high price, 3, when more players play.
In fact, as $ n \rightarrow \infty $, expected revenue goes to 3 in this scenario.

\end{enumerate}


\end{enumerate}

\end{document}
