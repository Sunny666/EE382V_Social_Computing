\documentclass[twoside]{article}
\usepackage{graphics}
\setlength{\oddsidemargin}{0.25 in}
\setlength{\evensidemargin}{-0.25 in}
\setlength{\topmargin}{-0.6 in}
\setlength{\textwidth}{6.5 in}
\setlength{\textheight}{8.5 in}
\setlength{\headsep}{0.75 in}
\setlength{\parindent}{0 in}
\setlength{\parskip}{0.1 in}

%
% The following commands set up the lecnum (lecture number)
% counter and make various numbering schemes work relative
% to the lecture number.
%
\newcounter{lecnum}
\renewcommand{\thepage}{\thelecnum-\arabic{page}}
\renewcommand{\thesection}{\thelecnum.\arabic{section}}
\renewcommand{\theequation}{\thelecnum.\arabic{equation}}
\renewcommand{\thefigure}{\thelecnum.\arabic{figure}}
\renewcommand{\thetable}{\thelecnum.\arabic{table}}

%
% The following macro is used to generate the header.
%
\newcommand{\lecture}[4]{
   \pagestyle{myheadings}
   \thispagestyle{plain}
   \newpage
   \setcounter{lecnum}{#1}
   \setcounter{page}{1}
   \noindent
   \begin{center}
   \framebox{
      \vbox{\vspace{2mm}
    \hbox to 6.28in { {\bf EE 382V: Social Computing
                        \hfill Fall 2018} }
       \vspace{4mm}
       \hbox to 6.28in { {\Large \hfill Lecture #1: #2  \hfill} }
       \vspace{2mm}
       \hbox to 6.28in { {\it Lecturer: #3 \hfill Scribe: #4} }
      \vspace{2mm}}
   }
   \end{center}
   \markboth{Lecture #1: #2}{Lecture #1: #2}
   %{\bf Disclaimer}: {\it These notes have not been subjected to the
   %usual scrutiny reserved for formal publications.  They may be distributed
   %outside this class only with the permission of the Instructor.}
   \vspace*{4mm}
}

%
% Convention for citations is authors' initials followed by the year.
% For example, to cite a paper by Leighton and Maggs you would type
% \cite{LM89}, and to cite a paper by Strassen you would type \cite{S69}.
% (To avoid bibliography problems, for now we redefine the \cite command.)
% Also commands that create a suitable format for the reference list.
\renewcommand{\cite}[1]{[#1]}
\def\beginrefs{\begin{list}%
        {[\arabic{equation}]}{\usecounter{equation}
         \setlength{\leftmargin}{2.0truecm}\setlength{\labelsep}{0.4truecm}%
         \setlength{\labelwidth}{1.6truecm}}}
\def\endrefs{\end{list}}
\def\bibentry#1{\item[\hbox{[#1]}]}

%Use this command for a figure; it puts a figure in wherever you want it.
%usage: \fig{NUMBER}{SPACE-IN-INCHES}{CAPTION}
\newcommand{\fig}[3]{
			\vspace{#2}
			\begin{center}
			Figure \thelecnum.#1:~#3
			\end{center}
	}
% Use these for theorems, lemmas, proofs, etc.
\newtheorem{theorem}{Theorem}[lecnum]
\newtheorem{lemma}[theorem]{Lemma}
\newtheorem{proposition}[theorem]{Proposition}
\newtheorem{claim}[theorem]{Claim}
\newtheorem{corollary}[theorem]{Corollary}
\newtheorem{definition}[theorem]{Definition}
\newenvironment{proof}{{\bf Proof:}}{\hfill\rule{2mm}{2mm}}

% **** IF YOU WANT TO DEFINE ADDITIONAL MACROS FOR YOURSELF, PUT THEM HERE:

\begin{document}
%FILL IN THE RIGHT INFO.
%\lecture{**LECTURE-NUMBER**}{**DATE**}{**LECTURER**}{**SCRIBE**}
\lecture{29}{November 10}{Vijay Garg}{Bryson Banks}
%\footnotetext{These notes are partially based on those of Nigel Mansell.}

% **** YOUR NOTES GO HERE:

% Some general latex examples and examples making use of the
% macros follow.  
%**** IN GENERAL, BE BRIEF. LONG SCRIBE NOTES, NO MATTER HOW WELL WRITTEN,
%**** ARE NEVER READ BY ANYBODY.
\section{Answers to Problems on Auctions}
The following two exercises focus on second-price, sealed-bid auctions where a seller is selling some object that bidders each have their own independent, private valuations for.

\subsection{Exercise 9-3}
(a) Four possible outcomes of two bidders having valuations 0 or 1:

\begin{itemize}
  \item $(v_{1}, v_{2}) = (0, 0)$: Bidder 1 or 2 wins and pays 0.
  \item $(v_{1}, v_{2}) = (0, 1)$: Bidder 2 wins and pays 0.
  \item $(v_{1}, v_{2}) = (1, 0)$: Bidder 1 wins and pays 0.
  \item $(v_{1}, v_{2}) = (1, 1)$: Bidder 1 or 2 wins and pays 1.
\end{itemize}

If we equally weight the probability of each of the four possible outcomes above, then the probability of each outcome is $\frac{1}{4}$. With this knowledge, we can then show the seller's expected revenue ($ER_{S}$) is also $\frac{1}{4}$.

$$ER_{S} = \frac{1}{4}(0) + \frac{1}{4}(0) + \frac{1}{4}(0) + \frac{1}{4}(1) = \frac{1}{4}$$

(b) Eight possible outcomes of three bidders having valuations 0 or 1:

\begin{itemize}
  \item $(v_{1}, v_{2}, v_{3}) = (0, 0, 0)$: Bidder 1, 2, or 3 wins and pays 0.
  \item $(v_{1}, v_{2}, v_{3}) = (0, 0, 1)$: Bidder 3 wins and pays 0.
  \item $(v_{1}, v_{2}, v_{3}) = (0, 1, 0)$: Bidder 2 wins and pays 0.
  \item $(v_{1}, v_{2}, v_{3}) = (0, 1, 1)$: Bidder 2 or 3 wins and pays 1.
  \item $(v_{1}, v_{2}, v_{3}) = (1, 0, 0)$: Bidder 1 wins and pays 0.
  \item $(v_{1}, v_{2}, v_{3}) = (1, 0, 1)$: Bidder 1 or 3 wins and pays 1.
  \item $(v_{1}, v_{2}, v_{3}) = (1, 1, 0)$: Bidder 1 or 2 wins and pays 1.
  \item $(v_{1}, v_{2}, v_{3}) = (1, 1, 1)$: Bidder 1, 2, or 3 wins and pays 1.
\end{itemize}

If we equally weight the probability of each of the eight possible outcomes above, then the probability of each outcome is $\frac{1}{8}$. With this knowledge, we can then show the seller's expected revenue ($ER_{S}$) has increased to $\frac{1}{2}$ with the additional bidder.

$$ER_{S} = \frac{1}{8}(0) + \frac{1}{8}(0) + \frac{1}{8}(0) + \frac{1}{8}(1) + \frac{1}{8}(0) + \frac{1}{8}(1) + \frac{1}{8}(1) + \frac{1}{8}(1) = \frac{4}{8} = \frac{1}{2}$$

(c) We see that going from two bidders to three bidders increased the seller's expected revenue. If we were to add in an additional fourth bidder, we would again see the seller's expected revenue increase. The same again with a fifth bidder, and a sixth. In fact, every increase in the total number of bidders will increase the seller's expected revenue because as the number of total bidders increases, so too does the probability that at least two bidders will bid 1, which is the only outcomes where the seller earns any revenue. It makes logical sense that adding bidders will only increase the chances of at least two of them bidding 1, but we also know that probability, while converging to 1, will never actually reach 1. Thus, we see the seller's expected revenue will similarly converge to 1 as the total number of bidders is increased, but never actually reach 1.

\subsection{Exercise 9-4}
(a) We know a always bids its valuation, but b sometimes makes a mistake with its bid. When b's actual valuation is 0, half the time it bids 1 mistakenly. Below we outline the possible outcomes of valuations and bids.

\begin{itemize}
  \item $(v_{a}, v_{b}) = (0, 0)$:
  \begin{itemize}
    \item $(b_{a}, b_{b}) = (0, 0)$: Bidder a or b wins and pays 0.
    \item $(b_{a}, b_{b}) = (0, 1)$: {\bf Mistake bid.} Bidder b wins and pays 0.
  \end{itemize}
  \item $(v_{a}, v_{b}) = (0, 1)$:
  \begin{itemize}
    \item $(b_{a}, b_{b}) = (0, 1)$: Bidder b wins and pays 0.
  \end{itemize}
  \item $(v_{a}, v_{b}) = (1, 0)$: Bidder 1 wins and pays 0.
  \begin{itemize}
    \item $(b_{a}, b_{b}) = (1, 0)$: Bidder a wins and pays 0.
    \item $(b_{a}, b_{b}) = (1, 1)$: {\bf Mistake bid.} Bidder a or b wins and pays 1.
  \end{itemize}
  \item $(v_{a}, v_{b}) = (1, 1)$:
  \begin{itemize}
    \item $(b_{a}, b_{b}) = (1, 1)$: Bidder a or b wins and pays 1.
  \end{itemize}
\end{itemize}

In order to determine if a's dominant strategy is still to always bid its true valuation, we will consider those case where a's true valuation is 0, and then those where its true valuation is 1.

When a's true valuation is 0, we see that when a bids that same value, a only has a $\frac{1}{8}$ chance to win the auction and pay its true valuation of 0. When it instead bids 1, it has a $\frac{1}{4}$ chance to win and pay its true valuation 0, but also has a $\frac{3}{8}$ chance to win and have to pay 1, which is above its valuation of 0. Thus, this is not advisable, and sticking with its true valuation still seems the dominant strategy.

When a's true valuation is 1, we see that when a bids that same value, a has a $\frac{5}{8}$ chance to win the auction and pay its true valuation of 1 or less. When it instead bids 0, its chances of winning drop to $\frac{1}{8}$. Thus, again we see that sticking with its true valuation remains the dominant strategy.

(b) If we equally weight the probability of each of the four possible valuations above at $\frac{1}{4}$, and equally the possible bids in each valuation case, we can then show the seller's expected revenue ($ER_{S}$) is $\frac{3}{8}$.

$$ER_{S} = \frac{1}{4}(\frac{1}{2}(0) + \frac{1}{2}(0)) + \frac{1}{4}(0) + \frac{1}{4}(\frac{1}{2}(0) + \frac{1}{2}(1)) + \frac{1}{4}(1) = \frac{1}{8} + \frac{1}{4} =  \frac{3}{8}$$



\end{document}





