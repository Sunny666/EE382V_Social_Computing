%
% This is the LaTeX template file for lecture notes for EE 382C/EE 361C.
%
% To familiarize yourself with this template, the body contains
% some examples of its use.  Look them over.  Then you can
% run LaTeX on this file.  After you have LaTeXed this file then
% you can look over the result either by printing it out with
% dvips or using xdvi.
%
% This template is based on the template for Prof. Sinclair's CS 270.

\documentclass[twoside]{article}
\usepackage{graphics}
\setlength{\oddsidemargin}{0.25 in}
\setlength{\evensidemargin}{-0.25 in}
\setlength{\topmargin}{-0.6 in}
\setlength{\textwidth}{6.5 in}
\setlength{\textheight}{8.5 in}
\setlength{\headsep}{0.75 in}
\setlength{\parindent}{0 in}
\setlength{\parskip}{0.1 in}

%
% The following commands set up the lecnum (lecture number)
% counter and make various numbering schemes work relative
% to the lecture number.
%
\newcounter{lecnum}
\renewcommand{\thepage}{\thelecnum-\arabic{page}}
\renewcommand{\thesection}{\thelecnum.\arabic{section}}
\renewcommand{\theequation}{\thelecnum.\arabic{equation}}
\renewcommand{\thefigure}{\thelecnum.\arabic{figure}}
\renewcommand{\thetable}{\thelecnum.\arabic{table}}

%
% The following macro is used to generate the header.
%
\newcommand{\lecture}[4]{
   %\pagestyle{myheadings}
   \thispagestyle{plain}
   \newpage
   %\setcounter{lecnum}{#1}
   %\setcounter{page}{1}
   \noindent
   \begin{center}
   \framebox{
      \vbox{\vspace{2mm}
    \hbox to 6.28in { {\bf EE 382V: Social Computing
                        \hfill Fall 2018} }
       \vspace{4mm}
       \hbox to 6.28in { {\Large \hfill Question 10.#1  \hfill} }
       \vspace{2mm}
       \hbox to 6.28in { {\it Lecturer: #3 \hfill Scribe: #4} }
      \vspace{2mm}}
   }
   \end{center}
   %\markboth{Lecture #1: #2}{Lecture #1: #2}
   %{\bf Disclaimer}: {\it These notes have not been subjected to the
   %usual scrutiny reserved for formal publications.  They may be distributed
   %outside this class only with the permission of the Instructor.}
   \vspace*{4mm}
}

%
% Convention for citations is authors' initials followed by the year.
% For example, to cite a paper by Leighton and Maggs you would type
% \cite{LM89}, and to cite a paper by Strassen you would type \cite{S69}.
% (To avoid bibliography problems, for now we redefine the \cite command.)
% Also commands that create a suitable format for the reference list.
\renewcommand{\cite}[1]{[#1]}
\def\beginrefs{\begin{list}%
        {[\arabic{equation}]}{\usecounter{equation}
         \setlength{\leftmargin}{2.0truecm}\setlength{\labelsep}{0.4truecm}%
         \setlength{\labelwidth}{1.6truecm}}}
\def\endrefs{\end{list}}
\def\bibentry#1{\item[\hbox{[#1]}]}

%Use this command for a figure; it puts a figure in wherever you want it.
%usage: \fig{NUMBER}{SPACE-IN-INCHES}{CAPTION}
\newcommand{\fig}[3]{
			\vspace{#2}
			\begin{center}
			Figure \thelecnum.#1:~#3
			\end{center}
	}
% Use these for theorems, lemmas, proofs, etc.
\newtheorem{theorem}{Theorem}[lecnum]
\newtheorem{lemma}[theorem]{Lemma}
\newtheorem{proposition}[theorem]{Proposition}
\newtheorem{claim}[theorem]{Claim}
\newtheorem{corollary}[theorem]{Corollary}
\newtheorem{definition}[theorem]{Definition}
\newenvironment{proof}{{\bf Proof:}}{\hfill\rule{2mm}{2mm}}

% **** IF YOU WANT TO DEFINE ADDITIONAL MACROS FOR YOURSELF, PUT THEM HERE:

\begin{document}
%FILL IN THE RIGHT INFO.
%\lecture{**LECTURE-NUMBER**}{**DATE**}{**LECTURER**}{**SCRIBE**}
\lecture{4}{August 24}{Vijay Garg}{Jason Trout}
%\footnotetext{These notes are partially based on those of Nigel Mansell.}

% **** YOUR NOTES GO HERE:

% Some general latex examples and examples making use of the
% macros follow.  
%**** IN GENERAL, BE BRIEF. LONG SCRIBE NOTES, NO MATTER HOW WELL WRITTEN,
%**** ARE NEVER READ BY ANYBODY.
Sellers a, b, and c are selling their houses for prices of 3, 1, and 0, respectively.\\\\
Buyer x values house a at 12, house b at 9, and house c at 8.\\
Buyer y values house a at 10, house b at 3, and house c at 6.\\
Buyer z values house a at 8, house b at 6, and house c at 5.\\
\\
Buyer x receives the maximum payoff by purchasing from Seller a.\\
\indent Payoff with a  = 12 - 3 = 9.\\\\
Buyer y receives the maximum payoff by purchasing from Seller a.\\
\indent Payoff with a = 10 - 3 = 7.\\\\
Buyer z receives the maximum payoff by purchasing from either Seller a, b, or c.\\
\indent Payoff with a = 8 - 3 = 5\\
\indent Payoff with b = 6 - 1 = 5\\
\indent Payoff with c = 5 - 0 = 5\\\\

\includegraphics{bipartite_graph1.png}\\\\
This set of prices is not market clearing because Buyers x and y both want to purchase from Seller a in order to maximize their payoffs. Seller a should raise their price in the next round of the bipartite auction procedure.

%FILL IN THE RIGHT INFO.
%\lecture{**LECTURE-NUMBER**}{**DATE**}{**LECTURER**}{**SCRIBE**}
\lecture{5}{August 24}{Vijay Garg}{Jason Trout}
%\footnotetext{These notes are partially based on those of Nigel Mansell.}

Sellers a, b, and c are selling their houses for prices 4, 3, and 1, respectively.\\\\
Buyer x values house a at 7, house b at 7, and house c at 4.\\
Buyer y values house a at 7, house b at 6, and house c at 3.\\
Buyer z values house a at 5, house b at 4, and house c at 3.\\
\\
Buyer x receives the maximum payoff by purchasing from Seller b.\\
\indent Payoff with b  = 7 - 3 = 4.\\\\
Buyer y receives the maximum payoff by purchasing from Sellers a or b.\\
\indent Payoff with a = 7 - 4 = 3.\\
\indent Payoff with b = 6 - 3 = 3.\\\\
Buyer z receives the maximum payoff by purchasing from Seller c.\\
\indent Payoff with c = 3 - 1 = 2\\\\
\includegraphics{bipartite_graph2.png}\\\\
This set of prices is market clearing. Buyer x can purchase from Seller b, Buyer y can purchase from Seller a, and Buyer z can purchase from Seller c.  Each buyer purchases from a unique seller and receives their maximum payoff.



\end{document}